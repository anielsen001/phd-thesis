% 30 May 2001 commenst from Fred
% 30 May 2001 comments from Paola

We used a scanning liquid-helium cooled 
superconducting quantum interference
device microscope to study the magnetic properties
of superconducting devices and thin films from $4.2\,\kelvin$ to 
$13\,\kelvin$. 
The microscope provides spatially resolved magnetic information
about samples and a new source of data to test 
existing theories of the distribution of magnetic field in
superconductors. 

In $\mathrm{Nb}-\mathrm{Al}_2\mathrm{O}_3-\mathrm{Nb}$ \jjas\ of differing
square lattice geometry ($30\times 100$ and $100 \times 150$)
we made field-cooled measurements of the magnetization as a function
of external field (from zero up to $3\,\Gauss$) and
found that the array is preferentially paramagnetic. 
This demonstrates that the observation of paramagnetism in a superconductor
does not provide evidence for an exotic symmetry in the order 
parameter.
Furthermore, we explain
this preferential paramagnetism
with a simple  model for flux screening which has applicability to 
a generic multiply-connected superconducting system. 

In \ybco\ films on 
Rolled Annealed Biaxially Textured Substrates 
(RABiTS\trademark)\footnote{RABiTS is a registered trademark of
Oak Ridge National Laboratory.} we found the spontaneous 
magnetization of the 
RABiTS substrates to be more complicated than described in 
previously published work.
Furthermore, we found YBCO films on RABiTS films to have small holes 
(diameter $<10\,\micron$) which we positively 
correlate to the spatial dependence of DC magnetic hysteresis
in the films. 
Additionally, we made detailed measurements 
of the magnetic field
distribution in the Meissner state (reversible magnetization) for
YBCO films on strontium titanate substrates. From
these measurements we determined the demagnetizing factor
to be $0.961$ and the 
first critical field to be 
less than $300\,\mOe$. Additionally, we infer the current
distribution which leads to this magnetic field distribution and 
demonstrate that it agrees with recent theoretical predictions. 
