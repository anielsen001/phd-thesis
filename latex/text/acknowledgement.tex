
\acknowledgements{

\setlength{\parindent}{0.25in}

\indent
Fred Cawthorne did a beautiful job of setting up the lab, and
organizing things (as well as amassing an unsurpassed collection of
resistors, capacitors and other miscellaneous electronic components). 
It is only possible to use the SQUID 
microscope today because of his hard work. The original designs
for the SQUID computer control were his and have provided a
great base to build upon. Fred Strauch worked a brief time 
in the sub-sub-basement made great contributions to the automation
of the scanning procedure by implementing GPIB control in the 
microscope controller. 

\indent
Prof. Rick Newrock, at the University 
of Cincinnati, has been very gracious in allowing periodic visits to 
the University of Cincinnati to discuss the microscope and the 
results and conclusions of PME measurements. I am greatly indebted to 
him for acknowledging my work in several presentations of his. 
I also need to thank Jenny Holzer, a student of Prof. 
Newrock, who provided several crucial pieces of knowledge required
to run the SQUID microscope. 

I am greatly indebted to Erin Fleet and Guy Chatraphorn for 
providing assistance and inspiration in working with and interpreting
the data from the SQUID microscope. Perhaps most importantly, it was
with their help that we developed the important sub-basement conservation
law for SQUIDs. 

I have to thank Brana Vasilic for many very useful conversations, 
particularly about the meaning of the quantum number $n$. Doug
Strachan has been a great friend down in the sub-sub-basement, and
as the only other member of the sub-sub-basement coffee club 
has been the source of many great ideas and fruitful conversations. 
Roberto Ramos provided many useful ideas for the work with the 
probe. Matt Kenyon was a great source of help, especially in the 
early stages with leak checking.

Of course, none of this would have been possible without the excellent help
from Jane Hessing who always made sure I was caught up with the university
bureaucracy. Becky Lorenz is a great welder and without her skilled hands
the SQUID microscope would have been in a state of perpetual leaks. 

Most importantly, I must thank Evelyn for putting up with
all of my crazy schemes and for graciously allowing me to make helium
transfers in the middle of many dates.

}
