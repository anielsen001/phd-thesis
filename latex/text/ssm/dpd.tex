% this is the digital phase detector description as written
% by Fred Cawthorne
% LaTeX and citations by Aaron Nielsen

\begin{quotation}

\subsubsection{Digital Phase Detector}

The key portion of the following circuit is the CD4066 quad bilateral switch%
\cite{harris_semiconductor}, 
connected as a switching type phase sensitive detector. 

System timing is derived from a $25.6\,\mathrm{MHz}$ 
crystal clock\cite{spk_electronics}.  The clock is 
divided by two presetable 8 bit dividers, type 74ALS\-867A%
\cite{texas_instruments}.  One of the 
two is preset at the terminal count and the other is programmable by a 
circuit board mounted DIP switch.  The $100\,\kHz$ 
output of the preset divider,
U2, is buffered by opamp U4B to allow external monitoring of the clock.

By applying these two phase adjustable $100\,\kHz$ clocks to two opposing 
quadrants of the detector, a DC error voltage may be derived during 
nonsynchronous periods in signal applied to the external input port.  
The input from the squid preamp is applied to 1 and 8 of the detector 
with reference to common at pins 4 and 11.  U4a amplifies the error 
voltage and damps spurious harmonics by the use of  C3.

To set up the proper phase relationship between clock quadrants:

\begin{enumerate}

\item 	Remove all AC bias from the squid.

\item	Set the experimental baseline state in the squid, normal or not.

\item	Set the shift code for U1 (using the DIP switch S1) so  that the 
error voltage output from the 
detector is zero.  Note that switch nine is a phase reversal switch.

\end{enumerate}

This completes the setup procedure.  Any phase shifts in the input 
signal from the normal state will produce a bipolar error voltage 
proportional to the phase error between the system clock and the 
squid signal.

\end{quotation}
