
\newcommand{\gpiblist}[2]{
 \parbox[t]{2.7in}{\renewcommand{\baselinestretch}{1.}
  \normalsize
  \sloppy #1 \\* \emph{Output}:#2 } 
}        

\newcommand{\rangelist}[1]{
 \parbox[t]{1in}{\renewcommand{\baselinestretch}{1.}
  \normalsize
  \sloppy #1}
}

\begin{longtable}{l|l|l}
\caption[\squidbox\ GPIB Commands]{
This is a listing of the GPIB commands accepted by the
\squidbox. The commands may be sent to the \squidbox, and
the appropriate action will be taken. If you query the \squidbox, it will
respond with the requested value.} \\ 
\hline   
\label{table:gpibcommands}
\textbf{Command} & \textbf{Description} & \textbf{Range}\\
\hline
\endfirsthead
\caption[]{\emph{GPIB commands continued}} \\
\hline
\textbf{Command} & \textbf{Description} & \textbf{Range} \\
\hline
\endhead
\hline
\multicolumn{2}{r}{\emph{continued on next page}}
\endfoot
\hline
\endlastfoot
``SCAND,I" &  \gpiblist{Set Scan Direction to I }{None} 
   &\rangelist{0=Horizontal\\ 1=Vertical} \\
``CRASHD,I" & \gpiblist{Set Crash Detection to I }{None} & 
  \rangelist{0=Off \\ 1=On}\\
``RPOTS,I"  & \gpiblist{Set to record pot positions to I }{None} &
\rangelist{0=Off \\ 1=On}\\
``SCANXV,X"  & \gpiblist{Set Xvelocity1 (forward scan) to X}{None} & \\
``SCANYV,Y"  & \gpiblist{Set Yvelocity1 (forward scan) to Y}{None}& \\
``SCANXA,X" & \gpiblist{Set Xaccel1 (forward scan) to X}{None}& \\
``SCANYA,Y" & \gpiblist{Set Yaccel1 (forward scan) to Y}{None}& \\
``BSCANXV,X" & \gpiblist{Set Xvelocity2 (back retrace) to X}{None}& \\
``BSCANYV,Y" & \gpiblist{Set Yvelocity2 (back retrace) to Y}{None}& \\
``BSCANXA,X" & \gpiblist{Set Xaccel2 (back retrace) to X}{None}& \\
``BSCANYA,Y" & \gpiblist{Set Yaccel2 (back retrace) to Y}{None}& \\
``NUMROWS,I" & \gpiblist{Set NumRows (data points per scan line) to I}{None}&\\
``NUMCOLS,I" & \gpiblist{Set NumColumns (number of scan lines) to I}{None}&\\
``STARTUL,X,Y" & \gpiblist{Set Start Upper Left Coordinates to X, Y}{None}&\\
``ENDLR,X,Y"   & \gpiblist{Set End Lower Right Coordinates to X,Y}{None}&\\
``*IDN?,"      & \gpiblist{Simple Identification Query}{``Model 1.0 SQUIDBOX"}&\\
``SCAND?,"     & \gpiblist{Scan\_Direction Query}{``Scan\_Direction"}&\\
``CRASHD?,"    & \gpiblist{Crash\_Detect Query}{``Crash\_Detect"}&\\
``RPOTS?,"     & \gpiblist{Record\_Pots Query}{``Record\_Pots"}&\\
``SCANXV?,"    & \gpiblist{Xvelocity1 Query}{``Xvelocity1"}&\\
``SCANYV?,"    & \gpiblist{Yvelocity1 Query}{``Yvelocity1"}&\\
``SCANXA?,"    & \gpiblist{Xaccel1 Query}{``Xaccel1"}&\\
``SCANYA?,"    & \gpiblist{Yaccel1 Query}{``Yaccel1"}&\\
``BSCANXV?,"   & \gpiblist{Xvelocity2 Query}{``Xvelocity2"}&\\
``BSCANYV?,"   & \gpiblist{Yvelocity2 Query}{``Yvelocity2"}&\\
``BSCANXA?,"   & \gpiblist{Xaccel2 Query}{``Xaccel2"}&\\
``BSCANYA?,"   & \gpiblist{Yaccel2 Query}{``Yaccel2"}&\\
``NUMROWS?,"   & \gpiblist{NumRows Query}{``NumRows"}&\\
``NUMCOLS?,"   & \gpiblist{NumColumns Query}{``NumColumns"}&\\
``STARTUL?,"   & \gpiblist{Start Upper Left Coordinates Query}{``StartULx,StartULy"}&\\
``ENDLR?,"     & \gpiblist{End Lower Right Query}{``EndLRx,EndLRy"}&\\
``XY?,"        & \gpiblist{Current Position Query}{``X,Y"}&\\
``PRETENSION," & \gpiblist{Pretension Command. This command is deprecated.
it is better to do this using the GOTOXY command. The pretension command moves
to Start X+6000,Y and may be dangerous to do if you are not careful.}{None}&\\
"START,"      & \gpiblist{Start Scanning}{``STARTSCAN" or ``NOSCAN"}&\\
``GOTOXY,X,Y'' & \gpiblist{Move SQUID to position X,Y}{None}&\\
``STATUS?''  & \gpiblist{Ask the controller what it's doing.}{1 if moving,
0 if doing nothing}&\\
\end{longtable}


The GPIB Primary Address is set to 6, but can be changed.
    Just redefine \CompVar{ADDRESS} at the top of the program.  

A file is created called \CompFile{gpibout.dat} which will allow 
    recording of status variables (\eg\ \CompVar{gpib\_readpending}, 
    \CompVar{gpib\_writepending}, \CompVar{gpib\_readbuf}, etc.) 
to assist in debugging.  

The scanning mode is somewhat complicated.
    The controller-\squidbox\ dialog is essentially the following 
    (for a single channel scan):\\
\begin{tabular}{l|l} 
\label{table:scanningdialog}   
    \textbf{Controller Command} & \textbf{\squidbox\ Response} \\ \hline
    \CompVar{"START,"}		& \CompVar{"STARTSCAN"} \\
    				& \CompVar{"FILES,1"} \\
    \CompVar{"FILE1,NAME?"}	& \CompVar{"data10.asc"}  \\
    \CompVar{"SEND,"} & \\	
\end{tabular}	
	
    At this point the interrupt routine is disabled on the \squidbox,
    and the \CompVar{ibwrtf-ibrdf} 
    file transfer is initiated.  The full sequence
    for a series of channels or potentiometer files continues
    the above sequence until all files are read.  
  
\subsubsection{Known Bugs}
\index{bugs,software!GPIB}
\label{GPIBbugs}

\index{bugs,software!GPIB!lock up after scan}
In GPIB mode the \squidbox\ will sometimes hang at the end of a 
scan, without responding to the GPIB controller. This can be worked
around by manually sending an arbitrary command to the \squidbox.
However, after sending the arbitrary command
we will have to manually copy the data files from
the \squidbox\ over to the GPIB controller. 

\index{bugs,software!GPIB!scan abort}
Also, aborting the scan might cause problems with \squidbox.
The error handling routines are not yet adequate.

