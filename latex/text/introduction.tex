% 23 May 2001 - with comments from Chris
% 30 May 2001 - with comments from Fred
% 30 May 2001 - comments from Paola
% 30 May 2001 - comments from R. Gomez
% 
% committee wants more discussion of the PME controversy in 
% the introduction

\chapter{Introduction}

\section{The Meissner effect}

\index{Meissner effect}
Superconductors are characterized by an extremely interesting
suite of magnetic properties. 
Meissner and Oschenfeld first described superconductors
as perfect diamagnets; 
a superconductor will develop
a magnetization to oppose any magnetic field applied to 
it \cite{meissner_dn_21_787_1933}. 
Screening currents which develop around the outside of the superconductor
cause this magnetization.
However, there is an upper bound on the amount of
current that a superconductor may carry, the critical current 
density $J_c$. When the current density in a superconductor reaches this 
critical current density, superconductivity ceases and the 
superconductor becomes a normal metal. A sufficiently large 
magnetic field, known as the critical field $H_c$,  
may be applied to cause a superconductor to become a normal metal.
In the simplest picture, this happens when the induced screening
currents exceed $J_c$. 


\index{vortex}
\index{\hcone}
\index{\hctwo}
\index{$H_c$}
It was later discovered that, for certain materials, a superconductor
would act as a Meissner diamagnet up to an external 
field \hcone\ but would then admit magnetic flux in quantized
units known as vortices. For $H> \hcone$ the superconductivity
is preserved until the external field increases to a still
higher second critical field, \hctwo, where the superconductivity is
destroyed. 
While in the Meissner state,
\ie\ below \hcone, the magnetization of the superconductor is
reversible. 

Above \hcone\ the magnetization is typically no longer 
reversible. This irreversibility arises due to
mechanisms that trap vortices in the superconductor, \ie\ pinning,
or some mechanism which provides a barrier to flux entry and exit, \ie\
the geometrical barrier \cite{zeldov_prl_73_1428_1994} or the 
Bean-Livingston barrier \cite{bean_prl_12_14_1973}.
Once vortices are introduced into the superconductor, the physical
properties change quite dramatically: the superconductor may not carry 
transport current without loss and there may be strong non-linearities
in the superconductor's AC response. 

\section{Types of superconductors}

\index{superconductor!type I}%
\index{superconductor!type II}%
Two generic categories describe superconductors: type I and type II. 
As discussed above, a type I superconductor does not allow vortices to 
penetrate and has only one critical field, $H_c$ while a 
type II superconductor
will allow vortices to penetrate and has two 
critical fields, \hcone\ and \hctwo.\footnote{for a more complete 
description, see Tinkham \cite{tinkham}, p.~134.}

\index{superconductor!\hightc}
\index{superconductor!\lowtc}

Superconductors can also be loosely categorized by their transition 
temperature. The \lowtc\ superconductors were known first and 
are often elemental, 
 \ie\ mercury, niobium, aluminum, etc., or metal alloys with $T_c$ up to 
about $24\,\kelvin$. 
The \hightc\ superconductors are the cuprate materials, which were
discovered in the 1980's. 
The cuprate superconductors may have transition temperatures
much higher than the \lowtc\ materials, up to about $150\,\kelvin$. 
The theory of Bardeen Cooper and Schreiffer 
(BCS) \cite{bardeen_pr_108_1957} 
explains the phenomena of superconductivity in \lowtc\
materials but seems to fail 
to adequately
describe the nature of superconductivity 
in the \hightc\ materials. 

The phe\-nom\-eno\-log\-i\-cal Ginzburg-Landau theory
\cite{ginzburg_zeitf_20_1064_1950} describes the normal 
metal-superconducting 
transition by means of a complex order parameter,
%
\begin{equation}
\label{eqn:orderparameter}
\Psi(\vec r) = \left| \Psi(\vec r) \right| \me^{\mi\varphi(\vec r)}.
\end{equation}
%
\index{BCS theory}
The original BCS theory said that the symmetry of $\Psi$ is spherical 
or of the so-called \swave\ type. 
\index{superconductor!\swave}
There are numerous proposed theories for \hightc\ superconductors, which
give different predictions for the symmetry of $\Psi(\vec r)$, so a measurement
of the symmetry of $\Psi(\vec r)$ can provide insights into the nature
of \hightc\ superconductivity. As will be discussed later in this
thesis, measurements of the magnetic properties may yield valuable 
clues into the symmetry of the order parameter. 


\section{Structure of the thesis}

This thesis is organized into chapters detailing different
aspects of my work. This introductory chapter is designed
to highlight what is well known about the magnetic properties
of superconductors and how that knowledge of magnetic properties
relates to what is known about the fundamental nature of
superconductivity. 
Chapter 2, \chapname{Josephson-junction arrays
as models of multiply connected superconductors}, discusses
why Josephson-junction arrays are particularly useful for
experimentally studying different magnetic phenomena, and describes
numerical models
for studying and interpreting magnetic phenomena
seen in arrays. 
Chapter 3, 
\chapname{Paramagnetic
Meissner effect in Josephson-junction arrays}, discusses our experimental 
observation of the paramagnetic Meissner effect using a scanning
SQUID microscope (SSM) to measure Josephson-junction arrays. 
Chapter 4, 
\chapname{A model of the
paramagnetic Meissner effect
in multiply-connected superconductors},
describes a model which
captures all of the important features of the paramagnetism in 
Josephson-junction arrays. This model is applicable to any 
multiply-connected superconductor. 
Chapter 5, 
\chapname{Magnetic penetration
into $\ybco$ thin films}, describes recent experiments using the 
SQUID microscope to observe the magnetic flux penetration into YBCO 
thin films, both in the Meissner state and in the critical state. 
Here we also discuss the special Rolled Annealed Biaxially 
Textured Substrate (RABiTS\trademark)\footnote{RABiTS is a registered 
trademark of Oak Ridge National Laboratory.} from Oak Ridge National Labs, 
and their potential applications for long superconducting 
wires. 
Chapter 6, \chapname{Prospective},
discusses some of the interesting phenomena which 
we observed, but chose not to pursue, in favor of the work discussed. 
The purpose of this
chapter is to explain to others what might be interesting and fruitful
future work. 

The appendix of this thesis consists primarily of a technical 
description of the experimental apparatus as well as the experimental
technique. This is meant to serve as a source of documentation 
for many of the parts of the experiment, particularly the 
computer program which controls the scanning mechanism and data
acquisition system. 

%\section{Electromagnetic unit systems}
%
%There are many different systems of units for dealing with 
%electromagnetic quantities. Throughout this thesis, we strive
%to use only MKS units for describing magnetic fields. The
%proper constitutive relationship for the magnetic field $\vec H$ and
%magnetic induction $\vec B$, including a magnetization $\vec M$ is
%
%\begin{equation}
%\vec B = \mu_0 (\vec H + \vec M)
%\end{equation}
%
%in which $\mu_0= 4 \pi \times 10^{-7} $ is the permeability of free space.
%The proper unit for $B$ is the Tesla and the proper unit for $H$ is

%For an excellent discussion of the relationship between different
%unit systems see Jackson\cite{jackson}, p.~811.


%\section{A word about terminology}
%
%There is a term to describe the magnetization
%of the sample after a large magnetic field has been applied
%and that field has been reduced to zero. 
%The proper term
%for this magnetization is ``remanence'' or ``remnant 
%magnetization.'' ``Remanence'' is a noun meant to describe 
%the zero field magnetization left behind; it would be
%redundant to speak of ``magnetic remanence.'' ``Remnant'' is
%an adjective which goes quite will with ``magnetization.''

